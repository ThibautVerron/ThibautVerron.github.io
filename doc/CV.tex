\documentclass{scrartcl}

\usepackage[utf8]{inputenc}

%% Serif font
% \IfFileExists{stix.sty}{%
%   \usepackage[upint]{stix}}%
% {\usepackage{lmodern}
%   \def\coloneq{:=}}

\usepackage{newtxtext}
%% Sans font
% \usepackage[scaled=0.875]{helvet}

%\usepackage[tt=false]{libertine}
% \usepackage{newtxtext}
\usepackage[T1]{fontenc}
%\renewcommand{\familydefault}{\sfdefault}

\usepackage{amsthm}
\usepackage[slantedGreek,cmintegrals,bigdelims,vvarbb,libertine,libaltvw,liby]{newtxmath}
%\useosf

\usepackage[english]{babel} 


\usepackage{url}
\usepackage[yyyymmdd]{datetime}
\renewcommand{\dateseparator}{-}
\usepackage{fancyhdr}
\usepackage{lastpage}

\pagestyle{fancy}
\fancyhf{}
\lfoot{Updated : \today}
\rfoot{Page \thepage{}/\pageref*{LastPage}}
\rhead{Thibaut \textsc{Verron}}
\renewcommand{\footrulewidth}{0.4pt}
\renewcommand{\headrulewidth}{0.4pt}

\fancypagestyle{first}{%
  \fancyhf{}% Clear header/footer
  \lfoot{Updated : \today}
  \rfoot{Page \thepage{}/\pageref*{LastPage}}
  \renewcommand{\footrulewidth}{0.4pt}
  \renewcommand{\headrulewidth}{0pt}
}

\usepackage{xspace}

\usepackage{tabu}

\usepackage{enumitem}

\usepackage[colorlinks]{hyperref}

\usepackage[backend=biber,maxnames=5]{biblatex}
\addbibresource{MyBiblio.bib}
\addbibresource{exposes.bib}

%\setkomafont{title}{\small}

\newcommand{\F}[1]{%                               % F algorithm
  \ifmmode%
  \mathsf{F_{#1}}%
  \else%
  \textsf{F}$_{\mathsf#1}$%
  \fi\xspace}
\newcommand{\FGLM}{FGLM\xspace}      

\newcommand{\pbox}[2][l]{%
  \renewcommand{\arraystretch}{1}%
  \begin{tabular}[t]{@{}#1@{}}#2\end{tabular}%
}

\newcommand{\NN}{\mathbb{N}}
\newcommand{\RR}{\mathbb{R}}
\newcommand{\QQ}{\mathbb{Q}}
\newcommand{\CC}{\mathbb{C}}

\title{}
\date{}

\begin{document}

\thispagestyle{first}

\newcommand{\structure}[1]{\textbf{\textsf{#1}}}

\noindent{\Huge \structure{Thibaut Verron}}\\[5pt]
\noindent{ \structure{Post-doctoral researcher, Johannes Kepler University}}\\[3pt]

% \begin{tabular}{ll}
%   \structure{Date de naissance} & 21 mars 1991 \\
%   \structure{Âge} & 25 ans \\
%   \structure{Adresse professionnelle} &
%   \pbox{2, rue Charles Camichel\\
%     Bureau F317\\
%     31000, Toulouse, France} \\
%   \structure{Adresse électronique} & thibaut.verron@irit.fr \\
%   \structure{Page web} & https://thibautverron.github.io
% \end{tabular}

\noindent\begin{minipage}[t]{0.4\linewidth}
  \structure{Date of birth:} 21 March 1991\\[3pt]
  \structure{Address:}\\
    Johannes Kepler University\\
    Altenbergerstraße 69,\\
    4040 Linz, Austria
\end{minipage}
%
\begin{minipage}[t]{0.6\linewidth}
  \structure{Citizenship:} French\\[3pt]
  \structure{E-mail:} \url{thibaut.verron@jku.at}\\[3pt]
  \structure{Webpage:} \url{https://thibautverron.github.io}
\end{minipage}
%\\[7pt]

% \begin{description}
%   % \item[Birth date (age)] 21 March 1991 (28)
%   % \item[Nationality] French
%   % \item[Address]
%   % \begin{tabular}[t]{l}
%   %   Johannes Kepler University\\
%   %   Altenbergerstraße 69,\\
%   %   4040 Linz/Donau, Austria
%   % \end{tabular}
% \end{description}

% \section*{Current position}
% \label{sec:Current-position}

% \begin{description}
%   \item[Status] Post-doctoral researcher
%   \item[Dates] October 2017 -- September 2019
%   \item[Location] Institute for Algebra, Johannes Kepler University (Linz, Austria)
%   \item[Project] Analytic and Enumerative Combinatorics (AEC, FWF 5004)
%   \item[Principal Investigator] Manuel Kauers
% \end{description}

\section*{Past and present research activities}

\begin{itemize}[itemsep=0pt]
  \item Gröbner bases for weighted homogeneous systems
  \item Strategy for the classification of real roots of determinantal systems, \\
  application to optimal control for medical imagery
  \item Signature-based Gröbner bases for polynomials over a ring
  \item Tropical Gröbner bases, Gröbner bases for Tate series
  \item D-finite functions
\end{itemize}

\section*{Employment}
\label{sec:Previous-positions}

\renewcommand{\arraystretch}{1.5}

\begin{tabular}{rl}
  \structure{2017 -- cur.} & \pbox{\structure{Post-doctoral researcher at JKU (Linz, Austria)}\\
    Supervisor: Manuel Kauers (Institute for Algebra)\\
    Keywords: computer algebra, algebraic combinatorics, D-finite functions% \\
    % Project: Algorithmic and Enumerative Combinatorics (AEC, FWF 5004)
  }\\
  \structure{2016 -- 2017} & \pbox{\structure{Post-doctoral researcher at INP-ENSEEIHT (Toulouse, France)}\\
  Supervisors: Joseph Gergaud, Olivier Cots  (Team \emph{Parallel algorithms}\emph{ and optimization})\\
  Keywords: optimal control
} 
\end{tabular}

\section*{Education}

\begin{tabular}{rl}
  \structure{2012 -- 2016} & \pbox{\structure{Ph.D thesis, University Pierre et Marie Curie (Paris, France)} \\
    Computer science
  } \\
  \structure{2011 -- 2012} & \pbox{\structure{Masters degree, University Paris-Sud 11 (Orsay, France)}\\
    Pure and Applied Mathematics, specialty Algebra, Analysis and Geometry} \\
  \structure{2009 -- 2013} & \pbox{\structure{École Normale Supérieure de Paris (France)}\\
    Diploma of the ENS, Major in Mathematics, minor in Computer Science} \\
  \structure{2007 -- 2009} & \structure{Preparatory classes MPSI, MP*, Lycée Hoche (Versailles, France)}
  \\
  \structure{2007} & \structure{A levels}
\end{tabular}


% \begin{description}
%   \item[2012 -- 2016] Thèse de doctorat, Université Pierre et Marie Curie (Paris)
%   \item[2011 -- 2012] Master 2, Université Paris-Sud 11 (Orsay)\\
%   Mathématiques fondamentales et appliquées, spécialité Algèbre, analyse et géométrie\\
%   Mention Assez Bien
%   \item[2009 -- 2013] École Normale Supérieure de Paris\\
%   Diplôme de l'ENS, spécialité Mathématiques, spécialité secondaire Informatique
%   \item[2007 -- 2009] Classes préparatoires MPSI, MP*, Lycée Hoche (Versailles)
%   \item[2007] Baccalauréat, mention Très Bien
% \end{description}

%\newpage
\section*{Ph.D. thesis}
\label{sec:Formation}

\begin{description}
  \item[Dates] September 2012 -- September 2016 (defense: 26 September 2016)
  \item[Location] PolSys team, LIP6, Université Pierre et Marie Curie (Paris, France)
  \item[Supervisors] Jean-Charles Faugère, Mohab Safey El Din
  \item[Title] Regularization of Gröbner basis computations for weighted and determinantal systems, and an application to medical imagery
  % \item[Defended on] 26 september 2016
  \item[Keywords] polynomial systems; Gröbner bases; structured systems; weighted-homogeneous systems; determinantal systems; real algebraic geometry
\end{description}

\paragraph{Committee}${}$\\

\begin{tabular}{lll}
  \structure{Director} & Jean-Charles Faugère & Research director, Inria \\
  \structure{Advisor} & Mohab Safey El Din & Professor, UPMC \\
  \structure{Reviewer} & Laurent Busé & Researcher, Inria, HdR \\
  \structure{Reviewer} & Bruno Salvy & Research director, Inria \\
  \structure{Examiner} & Bernard Bonnard & Professor, Université de Bourgogne \\
  \structure{Examiner} & Stef Graillat & Professor, UPMC
\end{tabular}

\section*{Publications and communications}
\label{sec:Publ-comm}

\subsection*{Conference papers}

\begin{itemize}
  \item \fullcite{CVV-2019-Tate}
  \item \fullcite{VVY18}
  \item \fullcite{BFJSV16}
  \item \fullcite{FSV13}
\end{itemize}

\subsection*{Journal papers}

\begin{itemize}
  \item \fullcite{FV2018}
  \item \fullcite{BCRV19}
  \item \fullcite{FSV15}
\end{itemize}

\subsection*{Preprints and submitted papers}
\label{sec:Prepr-subm-papers}

\begin{itemize}
  \item \fullcite{FV2019}
\end{itemize}

\subsection*{Posters}

\begin{itemize}
  \item \fullcite{poster-ISSAC19}
\end{itemize}

\subsection*{Conference talks}
\label{sec:Quelq-autr-expos}

\begin{itemize}
  \item \fullcite{talk-ISSAC19}
  \item \fullcite{talk-AAG19}
  \item \fullcite{talk-JNCF19}
  \item \fullcite{talk-ACA18}
  \item \fullcite{talk-SMAI17}
  \item \fullcite{talk-ISSAC16}
  \item \fullcite{talk-GdRMOA15}
  \item \fullcite{talk-JNCF14}
  \item \fullcite{talk-RDHL14}
  \item \fullcite{talk-ISSAC13}
  \item \fullcite{talk-JNCF13}
\end{itemize}

% \section*{Sélection d'exposés}
% \label{sec:Selection-dexposes}

\section*{Software}
\label{sec:Software}

\paragraph{Tate Algebras}
\begin{itemize}[itemsep=0pt]
  \item SageMath package for working with Tate series over
  \(\mathbb{Z}_p\) and \(\mathbb{Q}_p\)
  \item Implementation of algorithms presented at ISSAC 2019
  \item Distributed with SageMath since version 8.5 (22/12/2018)
  \item Joint development with X.~Caruso and T.~Vaccon
  \item Link: \url{https://doc.sagemath.org/html/en/reference/power_series/sage/rings/tate_algebra.html}
\end{itemize}

\paragraph{Signature Gröbner bases over PIDs}
\begin{itemize}[itemsep=0pt]
  \item Toy implementation in Magma of signature-enabled versions of Möller's algorithms for computing Gröbner bases over PIDs
  \item Implementation of algorithms presented at ACA 2018 and SIAM AG 2019
%  \item Joint work with M.~Francis
  \item Link: \url{https://github.com/ThibautVerron/SignatureMoller}
\end{itemize}

\paragraph{Real algebraic classification algorithms for determinantal varieties}

\begin{itemize}[itemsep=0pt]
  \item Implementation in Maple of algorithms computing a classification of the real singularities of determinantal varieties
  \item Implementation of algorithms presented at ISSAC 2016
  \item Example run on an application to optimal control in medical imagery
  \item Joint development with M.~Safey El Din
  \item Link: \url{http://mercurey.gforge.inria.fr}
\end{itemize}

\section*{Teaching and supervising experience}
\label{sec:Teaching}

% \paragraph{2018 -- 2019 : Guest lecturer / teaching assistant in Mathematics, JKU, Linz (Austria)}
% \begin{itemize}
%   \item \structure{Co-advisor for a bachelor thesis}, together with Manuel Kauers (in progress)
%   \item \structure{Special lecture}: \emph{Computer Algebra 2} \\
%   (Accessible from bachelor level, 15 lectures, 30h)\\
%   Preparation of lecture notes, final evaluation on programming exercises as homework
%   \item \structure{Exercise sessions}: \emph{Linear Algebra for computer scientists} (in progress) \\
%   (Bachelor level, 30h)\\
%   Grading of finals
% \end{itemize}
% \paragraph{2016 -- 2017 : Teaching assistant in Applied Mathematics, INP Toulouse (France)}
% \begin{itemize}
%   \item \structure{Programming sessions}: \emph{Ordinary Differential Equations} (Python with Scipy, Matlab)\\
%   (Bachelor level, 26h)\\
%   Evaluation of mini-projects (based on a short interview and written report)
% \end{itemize}
% \paragraph{2013 -- 2016 : Teaching assistant in Computer Science, UPMC, Paris (France)}
% \begin{itemize}
%   \item \structure{Exercise and programming sessions}: \emph{Working environment}, \emph{Databases}\\
%   (Bachelor level, 129h)\\
%   Participation in the preparation of the exams, setup of a framework for automated correction of the homework and exams, grading of homework throughout the semester and of finals, preparation and grading of short written tests
%   \item \structure{Programming sessions}: \emph{Introduction to programmation} (Python), \emph{Scientific computing}~(C), \emph{Computer Architecture} (Asm) \\
%   (Bachelor level, 60h)\\
%   Grading of homework throughout the semester
% \end{itemize}

\begin{tabular}{rl}
  {\structure{2018 -- 2019}} & \pbox{Guest lecturer / teaching assistant in Mathematics, JKU, Linz (Austria)\\
    Supervision of a bachelor thesis with M.~Kauers}\\
{\structure{2016 -- 2017}} & Teaching assistant in Applied Mathematics, INP Toulouse (France) \\ 
{\structure{2013 -- 2016}} & Teaching assistant in Computer Science, UPMC, Paris (France)
\end{tabular}



% \section*{Enseignement}
% \label{sec:Enseignement-2}

% \noindent Moniteur à l'UFR 919 (Ingénierie) de l'UPMC de 2013 à 2016:
% \begin{itemize}
%   \item 3 groupes de TD (118.5 heures équivalent TD) en licence d'informatique 2\ieme année
%   \item 3 groupes de TME (59.25 heures équivalent TD) en licence d'informatique 1\iere et 2\ieme année
% \end{itemize}


% \begin{tabular}{rl}
%   \structure{2018 -- 2019} & \structure{Guest lecturer / teaching assistant in Mathematics, JKU, Linz (Austria)} \\
%   & Special lecture on Computer Algebra (Bachelor to graduate level, 30h)\\
%   & Exercise sessions on Linear Algebra for computer scientists (Bachelor level, 30h)\\
%   \structure{2016 -- 2017} & \structure{Teaching assistant in Applied Mathematics, INP Toulouse (France)} \\
%   & Programming sessions (Scipy, Matlab) on Ordinary Differential Equations (Bachelor level, 26h) \\
%   \structure{2013 -- 2016} & \structure{Teaching assistant in Computer Science, UMPC, Paris (France)} \\
%   & Exercise and programming sessions: \emph{Working environment}, \emph{Databases} (Bachelor level, 120h)\\
%   & Programming sessions on \emph{Introduction to programmation} (Python), \emph{Scientific computing} (C), \emph{Computer Architecture} (Asm) (Bachelor level, 60h)\\
% \end{tabular}


% \begin{tabular}{rl}
%   \structure{2018 -- 2019} & JKU, Linz, Mathematics (special lecture, exercise sessions, total: 60h) \\
%   \structure{2016 -- 2017} & INP Toulouse, Applied Mathematics (programming sessions, total: 26h) \\
%   \structure{2013 -- 2016} & UPMC, Paris, Computer Science (exercise and programming sessions, total: 189h)
% \end{tabular}

\section*{Additional activities}

\begin{itemize}
  \item \structure{Software presentation award committee} for the International Symposium on Symbolic and Algebraic Computation (ISSAC) 2019
  \item \structure{Poster chair} for the 6th International Congress on Mathematical Software (ICMS), 2018
  \item \structure{Reviewer} for SODA, JSC, FPSAC, MACIS
\end{itemize}

\section*{Other information}

\begin{itemize}
  \item \structure{Languages}: French (native), English (fluent), German (advanced), Russian (basic), \\Turkish (basic)
  \item \structure{Programming languages}: Python, C, C++, OCaml, Haskell, Bash, Emacs lisp
  %\item \structure{Computer algebra systems}: Sage, Magma, Maple
  %\item \structure{Leasure activities}: Competitive programming (Prologin, Codingame, Google code jam...),
  % Inline speed-skating
\end{itemize}

% \section*{Language skills}
% \begin{itemize}
%   \item \structure{French}: Native speaker
%   \item \structure{English}: Fluent
%   \item \structure{German}: Upper intermediate
%   \item \structure{Russian}: Basic
% \end{itemize}

\newpage
\section*{References}

\newcommand{\reference}[3]{%
  \textbf{#1}, #2\\%
  E-mail: \url{#3}
}

\begin{itemize}
  \item \reference{J.-C.~Faugère}{research director, Inria Paris, France}{jean-charles.faugere@inria.fr}
  \item \reference{M.~Safey El Din}{professor, Sorbonne Universités, Paris, France}{mohab.safey@lip6.fr}
  \item \reference{M.~Kauers}{professor, Johannes Kepler University, Linz, Austria}{manuel@kauers.de}
  \item \reference{B.~Bonnard}{professor, Université de Bourgogne, Dijon, France}{bernard.bonnard@u-bourgogne.fr}
  \item \reference{X.~Caruso}{research director, CNRS, Université de Bordeaux, Bordeaux, France}{xavier.caruso@normalesup.org}
\end{itemize}

\end{document}


%%% Local Variables:
%%% mode: latex
%%% TeX-master: t
%%% End:
