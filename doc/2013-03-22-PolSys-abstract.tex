\documentclass[10pt,a4paper]{scrartcl}

\usepackage[utf8]{inputenc}  %% les accents dans le fichier.tex
%\usepackage{mathptmx}
\usepackage{lmodern}
\usepackage[T1]{fontenc}       %% Pour la césure des mots accentués

\usepackage{amsmath,amssymb}

\begin{document}
% Computing Gröbner bases for quasi-homogeneous systems
% Thibaut Verron

Let $K$ be a field and $(f_1, \ldots, f_n)\subset K[X_1, \ldots, X_n]$ be a sequence of quasi-homogeneous polynomials of respective weighted degrees $(d_1, \ldots, d_n)$.  By this, we mean that there exists $(w_{1},\dots,w_{n}) \in \mathbb{Z}_{>0}^{n}$ s.t. for any $1 \leq j \leq n$, the polynomial $f_{i}(X_{1}^{w_{1}},\dots,X_{n}^{w_{n}})$ is homogeneous and has degree $d_{i}$.

In this talk, we show how we can adapt the existing strategies for homogeneous systems to quasi-homogeneous systems. We also show that for generic quasi-homogeneous systems, the complexity of a Gröbner basis computation for a quasi-homogeneous system is polynomial in the weighted Bézout bound $\prod_{i=1}^n\frac{d_i}{w_i}$.

Joint work with Jean-Charles Faugère and Mohab Safey El Din

% This paper focuses on designing strategies for computing Gröbner bases of such systems.

% Under some genericity assumptions on the coefficients of the $f_i$'s, we prove the bound $\left (\sum_{i=1}^n(d_i-w_i)\right )+\max\{w_i\}$ on the degree of regularity of such systems. This is done by using well-known results on Hilbert series and designing a dedicated quasi-homogeneous matrix-F5 algorithm taking care of this structure. We provide a detailed complexity analysis of this algorithm, and we prove that we can divide the complexity bounds by $P^{\omega}$, where $P = \prod_{i=1}^{n}w_{i}$. We also show how to use the FGLM (change of ordering) step in this special setting to obtain a Gr\"obner basis for the lexicographical ordering in time polynomial in the weighted B\'ezout bound $\prod_{i=1}^n\frac{d_i}{w_i}$.

% All in all, we provide a complete computational strategy for solving quasi-homogeneous systems in time polynomial in this weighted Bézout bound. We finally prove that these complexity results apply for $0$-dimensional affine systems which are obtained by specializing to $1$ a variable in a quasi-homogeneous system. We provide some experiments showing that this computational strategy sometimes offer gains of up to 10, and allows to handle systems which cannot be tackled with previous approaches.

\end{document}
%%% Local Variables: 
%%% mode: latex
%%% TeX-master: t
%%% ispell-dictionary: "english"
%%% End: 
